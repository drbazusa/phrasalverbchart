\documentclass{article}

\usepackage[a4paper, margin=2cm]{geometry}
\usepackage[latin1]{inputenc}
\usepackage{tikz}
\usetikzlibrary{shapes,arrows}
\begin{document}
\pagestyle{empty}


% Define block styles
\tikzstyle{startstop} = [rectangle, rounded corners, 
minimum width=3cm, 
minimum height=1cm,
text centered, 
draw=black, 
fill=red!30]

\tikzstyle{io} = [trapezium, 
trapezium stretches=true, % A later addition
trapezium left angle=70, 
trapezium right angle=110, 
minimum width=3cm, 
minimum height=1cm, text centered, 
draw=black, fill=blue!30]

\tikzstyle{process} = [rectangle, 
minimum width=3cm, 
minimum height=1cm, 
text centered, 
text width=3cm, 
draw=black, 
fill=orange!30]

\tikzstyle{decision} = [circle, 
minimum width=3cm, 
minimum height=1cm, 
text centered, 
draw=black, 
fill=green!30]
\tikzstyle{arrow} = [thick,->,>=stealth]

\section*{Phrasal Verb Chart}

Look at your term in question. Determine if there is an object or pronoun present. Follow the chart to determine if it is separable or transitive.

\subsection*{License}
Phrasal Verb Chart \textcopyright  2024 by Dennis Russell Bovell is licensed under Creative Commons Zero v1.0 Universal

\begin{figure}[!ht]
  \centering
  \begin{tikzpicture}[node distance = 2cm]
    % nodes
    \node [startstop, anchor=west] (vp) {\Large verb + particle};
    \node [startstop, below of=vp, yshift=-3cm] (vpo) {\Large verb + particle + object};
    \node [startstop, below of=vpo, yshift=-3cm] (vop) {\Large verb + object + particle};
    \node [startstop, below of=vop, yshift=-3cm] (vrp) {\Large verb + pronoun + particle};
    
    \node [process, right of=vp, xshift=3cm] (insep) {\Large inseparable};
    \node [decision, right of=insep, xshift=3cm] (intrans) {\Large intransitive};


    \node [process, right of=vpo, xshift=3cm] (insepo) {\Large possibly inseparable, used as such};
    \node [process, right of=vop, xshift=3cm] (sepo) {\Large separable};
    \node [process, right of=vrp, xshift=3cm] (asepo) {\Large always separable};
    \node [decision, right of=vop, xshift=8cm] (transo) {\Large transitive};
    
    % edges
    \draw [arrow] (vp) -- (insep);
    \draw [arrow] (insep) -- (intrans);

    \draw [arrow] (vpo) -- (insepo);
    \draw [arrow] (vop) -- (sepo);

    \draw [arrow] (vrp) -- (asepo);
    \draw [arrow] (asepo) -| (transo);
    
    \draw [arrow] (sepo) -- (transo);
    \draw [arrow] (insepo) -| (transo);
    
    
  \end{tikzpicture}
\end{figure}

\end{document}